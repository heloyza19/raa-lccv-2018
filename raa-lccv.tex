\documentclass[raa-page]{lccv18}
%------------------------------------------------------------------------------
% LaTeX TEMPLATE FOR RAA 2018
%------------------------------------------------------------------------------
% DO NOT RENAME THIS FILE !!!!!!!

% REMARKS:
%------------------------------------------------------------------------------
% - USE AT LEAST 1 (ONE) AND AT MOST 2 (TWO) FIGURES
% - USE, PREFERENTIALLY, VECTOR GRAPHICS (PDF) INSTEAD OF RASTER GRAPHICS (JPEG)
%   OR PNG-IMAGES WITH TRANSPARENCY IN HIGH QUALITY
% - THE DOCUMENT MUST BE GENERATED ON A SINGLE PAGE


% FILL IN BELOW WITH YOUR WORK INFORMATION
\begin{document}
    
    % TITLE (DO NOT USE UPPERCASE)
    %------------------------------------------------------
    \settitle{Development of a program capable of simulating the dynamics of rigid bodies in two dimensions}

    % AUTHORS' INFORMATION
    %------------------------------------------------------
    \setmainauthor{Heloyza Helena da Silva Gois}
    \setmainauthoremail{heloyza19@gmail.com}
    %\setotherauthor{Ed}
   
    
    % TYPE OF WORK
    %------------------------------------------------------
    %\settype{Research \& Development Project}
    % \settype{Research \& Development Activity}
    \settype{Research Beginner}
    % \settype{Undergraduate Thesis}
    % \settype{Master's Thesis}
    % \settype{Doctoral Thesis}

    % DURATION AND STATUS
    %------------------------------------------------------
    \setduration{3 months}
    \setstatus{In progress}
    % \setstatus{Completed}

    % COURSE (REQUIRED FOR STUDENTS)
    %------------------------------------------------------
    % \setcourse{Civil Engineering}
     \setcourse{Chemical Engineering}
    % \setcourse{Petroleum Engineering}
    % \setcourse{Computer Science}

    % PARTNERS: CONTRIBUTOR LABS OR GROUPS (OPTIONAL)
    %------------------------------------------------------
    % \setpartners{CENPES/PETROBRAS}
    % \setpartners{LABGEO/UFAL}
    % \setpartners{IPT/USP}

    % FUNDING AGENCIES (OPTIONAL)
    %------------------------------------------------------
    \setfunding{CENPES/PETROBRAS}
    % \setfunding{CNPq}
    % \setfunding{FAPEAL}


    % INTRODUCTION TEXT AND FIGURE
    %------------------------------------------------------
    \setintro{
    Submerged structures, used in the hydrocarbon exploration process, are often located in regions where landslides are a potential risks. Thus, the study of the impact of these landslides on the submerged structures is essential to the design of the underwater equipments. In order to solve this problem, we will consider, initially, that the structures are rigid bodies with polygonal sections. This will allow the prediction of the body dynamics and the forces acting on it. In this context, the objective of this work is to develop a program capable of simulating the dynamics of rigid bodies of arbitrary geometry in two dimensions, that will be added later to the main program of the project.
}
    %------------------------------------------------------
  

    % METHODOLOGY TEXT AND FIGURE
    %------------------------------------------------------
    \setmethodology{
  The development of a program capable of simulating the dynamics of rigid bodies uses some simplifications in order to solve this problem: we use the pinball algorithm to solve the contact problem and the time is discretized using the traditional Euler algorithm. As, in the real problem, the bodies are not actually rigid, we also have an error associated with this approximation, i.e., the structures do not deform when forces are applied to them. Furthermore, applied forces creates only rotational and translational movements in the object. Thus, after determining the center of mass and moment of inertia of each object, knowing the acting forces (external and contact forces), the resulting force and torque are calculated. This allows the calculation of the entire dynamics of the object, with the aid of a numerical integrator to solve the Euler's equations of motion.\\
  To take the contact of different structures in consideration, the Pinball algorithm is used. In this algorithm, the edges of the rigid bodies are filled with discrete elements. These elements acts as contact units, transferring forces between rigid bodies and solving the contact problem.
}
    %------------------------------------------------------
  
    % RESULTS TEXT AND FIGURE
    %------------------------------------------------------
    \setresults{
The computational techniques described in the methodology were implemented in MATLAB, generating a computational module capable of solving the dynamics and contacts of rigid bodies. For the validation of the program, the energy conservation and the conservation of the linear and angular moments of the system were analyzed. A good approximation was obtained in the simulation of the objects dynamics, showing that the physical concepts involved were correctly implemented. Moreover, the smaller the increment of time used in the temporal integration, the better the results of the simulation, as expected.
}
    %------------------------------------------------------
     \setresultsfigure{ENERGIAS.jpg}{width=8cm}

    
    % DO NOT EDIT HERE !!!!!!!
    %------------------------------------------------------
    \createRAApage

\end{document}
