\documentclass[raa-page]{lccv18}
%------------------------------------------------------------------------------
% LaTeX TEMPLATE FOR RAA 2018
%------------------------------------------------------------------------------
% DO NOT RENAME THIS FILE !!!!!!!

% REMARKS:
%------------------------------------------------------------------------------
% - USE AT LEAST 1 (ONE) AND AT MOST 2 (TWO) FIGURES
% - USE, PREFERENTIALLY, VECTOR GRAPHICS (PDF) INSTEAD OF RASTER GRAPHICS (JPEG)
%   OR PNG-IMAGES WITH TRANSPARENCY IN HIGH QUALITY
% - THE DOCUMENT MUST BE GENERATED ON A SINGLE PAGE


% FILL IN BELOW WITH YOUR WORK INFORMATION
\begin{document}
    
    % TITLE (DO NOT USE UPPERCASE)
    %------------------------------------------------------
    \settitle{Development of a program capable of simulating the dynamics of rigid bodies in two dimensions}

    % AUTHORS' INFORMATION
    %------------------------------------------------------
    \setmainauthor{Heloyza Helena da Silva Gois}
    \setmainauthoremail{heloyza19@gmail.com}
    %\setotherauthor{Ed}
   
    
    % TYPE OF WORK
    %------------------------------------------------------
    %\settype{Research \& Development Project}
    % \settype{Research \& Development Activity}
    \settype{Research Beginner}
    % \settype{Undergraduate Thesis}
    % \settype{Master's Thesis}
    % \settype{Doctoral Thesis}

    % DURATION AND STATUS
    %------------------------------------------------------
    \setduration{3 months}
    \setstatus{In progress}
    % \setstatus{Completed}

    % COURSE (REQUIRED FOR STUDENTS)
    %------------------------------------------------------
    % \setcourse{Civil Engineering}
     \setcourse{Chemical Engineering}
    % \setcourse{Petroleum Engineering}
    % \setcourse{Computer Science}

    % PARTNERS: CONTRIBUTOR LABS OR GROUPS (OPTIONAL)
    %------------------------------------------------------
    % \setpartners{CENPES/PETROBRAS}
    % \setpartners{LABGEO/UFAL}
    % \setpartners{IPT/USP}

    % FUNDING AGENCIES (OPTIONAL)
    %------------------------------------------------------
    \setfunding{CENPES/PETROBRAS}
    % \setfunding{CNPq}
    % \setfunding{FAPEAL}


    % INTRODUCTION TEXT AND FIGURE
    %------------------------------------------------------
    \setintro{
    The submerged structures used in the hydrocarbon exploration process are subject to underwater landslides; which makes it essential to study the impact of these landslides on the structures, through the simulation of this problem.\\
    It is necessary to make simplifications of the problem in order to simulate the impact of underwater landslides on structures, with two-dimensional analysis, simplifications such as considering all matter as rigid bodies with polygonal sections.\\
    In view of the presented context, the objective is to develop a program capable of simulating the dynamics of rigid bodies of arbitrary geometry in two dimensions, so that it is possible to analyze the presented problem in a simplified way.
}
    %------------------------------------------------------
  

    % METHODOLOGY TEXT AND FIGURE
    %------------------------------------------------------
    \setmethodology{
  The development of the program capable of simulating the dynamics of rigid bodies uses the pinball model, making approximations to simulate the problem, such as considering all matter as rigid bodies with polygonal sections. In considering objects as rigid bodies, there is the simplification of considering that bodies do not deform when forces are applied to them; in addition, the applied forces create rotational and translational movements in the object. Thus, after determining the center of mass and moment of inertia of each object, knowing the acting forces (external and contact forces), the resulting force and torque are calculated to be able to simulate the entire dynamics of the object through computational techniques.\\
  The Pinball counting model consists of considering the edges of the polygon filled by balls of very small radius, and determining the contacts and forces acting through these balls, applying the law of Force-Displacement of the Discrete Element  Method (DEM).
}
    %------------------------------------------------------
  
    % RESULTS TEXT AND FIGURE
    %------------------------------------------------------
    \setresults{
 Through the computational techniques cited and the application of physical concepts, it was possible to develop a program, using MATLAB, that simulated the dynamics of objects in two dimensions. For the validation of the program, the energy conservation and the conservation of the linear and angular moments of the system were analyzed. It was obtained a good approximation in the simulation of the dynamics of the objects, that shows fidelity to the physical concepts involved, however the smaller the increment of time used in the temporal integration, the better the results of the simulation.
}
    %------------------------------------------------------
     \setresultsfigure{ENERGIAS.jpg}{width=8cm}

    
    % DO NOT EDIT HERE !!!!!!!
    %------------------------------------------------------
    \createRAApage

\end{document}
