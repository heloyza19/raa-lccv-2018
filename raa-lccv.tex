\documentclass[raa-page]{lccv18}
%------------------------------------------------------------------------------
% LaTeX TEMPLATE FOR RAA 2018
%------------------------------------------------------------------------------
% DO NOT RENAME THIS FILE !!!!!!!

% REMARKS:
%------------------------------------------------------------------------------
% - USE AT LEAST 1 (ONE) AND AT MOST 2 (TWO) FIGURES
% - USE, PREFERENTIALLY, VECTOR GRAPHICS (PDF) INSTEAD OF RASTER GRAPHICS (JPEG)
%   OR PNG-IMAGES WITH TRANSPARENCY IN HIGH QUALITY
% - THE DOCUMENT MUST BE GENERATED ON A SINGLE PAGE


% FILL IN BELOW WITH YOUR WORK INFORMATION
\begin{document}
    
    % TITLE (DO NOT USE UPPERCASE)
    %------------------------------------------------------
    \settitle{Development of a program capable of simulating the dynamics of rigid bodies in two dimensions}

    % AUTHORS' INFORMATION
    %------------------------------------------------------
    \setmainauthor{Heloyza Helena da Silva Gois}
    \setmainauthoremail{heloyza19@gmail.com}
    %\setotherauthor{Ed}
   
    
    % TYPE OF WORK
    %------------------------------------------------------
    %\settype{Research \& Development Project}
    % \settype{Research \& Development Activity}
     %\settype{Research Beginner}
    % \settype{Undergraduate Thesis}
    % \settype{Master's Thesis}
    % \settype{Doctoral Thesis}

    % DURATION AND STATUS
    %------------------------------------------------------
    \setduration{ months}
    \setstatus{In progress}
    % \setstatus{Completed}

    % COURSE (REQUIRED FOR STUDENTS)
    %------------------------------------------------------
    % \setcourse{Civil Engineering}
     \setcourse{Chemical Engineering}
    % \setcourse{Petroleum Engineering}
    % \setcourse{Computer Science}

    % PARTNERS: CONTRIBUTOR LABS OR GROUPS (OPTIONAL)
    %------------------------------------------------------
    % \setpartners{CENPES/PETROBRAS}
    % \setpartners{LABGEO/UFAL}
    % \setpartners{IPT/USP}

    % FUNDING AGENCIES (OPTIONAL)
    %------------------------------------------------------
    \setfunding{CENPES/PETROBRAS}
    % \setfunding{CNPq}
    % \setfunding{FAPEAL}


    % INTRODUCTION TEXT AND FIGURE
    %------------------------------------------------------
    \setintro{The Pinball algorithm is used for the numerical simulation of particle dynamics. It consists of approaching each particle by a polygonal object, the edges of the polygon are filled by balls of very small radius, in which these balls are used to make the contact detection between particles, so that later the acting forces are determined.The contact forces are determined from the penetration between the balls of the edge, through a law of contact that relates penetration and force.\\
    In order to develop a program capable of simulating the dynamic two-dimensional rigid bodies, the Pinball algorithm was used, considering the particles as rigid bodies, through MATLAB.}
    %------------------------------------------------------
  

    % METHODOLOGY TEXT AND FIGURE
    %------------------------------------------------------
    \setmethodology{In order to simulate the dynamics of rigid bodies in two dimensions, the program was developed through MATLAB so that, first, the center of mass and the moment of inertia of each rigid bodies were determined\\
    Then, the contact verification between the bodies is performed, using the pinball algorithm, which consists of considering the edges of the object filled with small balls, thus using these balls to determine the contact between the body, the search for contacts between the bodies is performed through the direct mapping of the balls.\\
  After determining some collision, the contact forces are calculated according to the Force-Displacement Law of the Discrete Element Method.\\
Thus, knowing the acting forces (external and contact forces), it was possible to simulate all the dynamics of the object through computational techniques.
    }
    %------------------------------------------------------
  

    % RESULTS TEXT AND FIGURE
    %------------------------------------------------------
    \setresults{
 Through the computational techniques cited and the application of physical concepts, it was possible to develop a program, using MATLAB, that simulated the dynamics of objects in two dimensions. For the validation of the program, the energy conservation and the conservation of the linear and angular moments of the system were analyzed. It was obtained a good approximation in the simulation of the dynamics of the objects, that shows fidelity to the physical concepts involved, however the smaller the increment of time used in the temporal integration, the better the results of the simulation.
}
    %------------------------------------------------------
     \setresultsfigure{ENERGIAS.jpg}{width=8cm}

    
    % DO NOT EDIT HERE !!!!!!!
    %------------------------------------------------------
    \createRAApage

\end{document}
